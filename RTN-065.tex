\documentclass[OPS,authoryear,toc]{lsstdoc}
\input{meta}

% Package imports go here.

% Local commands go here.

%If you want glossaries
%\input{aglossary.tex}
%\makeglossaries

\title{Initial USDF throughput tests}

% Optional subtitle
% \setDocSubtitle{A subtitle}

\author{%
William O'Mullane
}

\setDocRef{RTN-065}
\setDocUpstreamLocation{\url{https://github.com/lsst/rtn-065}}

\date{\vcsDate}

% Optional: name of the document's curator
% \setDocCurator{The Curator of this Document}

\setDocAbstract{%
This is a first attempt at using the new DM Camera interface for transferring images to USDF.
}

% Change history defined here.
% Order: oldest first.
% Fields: VERSION, DATE, DESCRIPTION, OWNER NAME.
% See LPM-51 for version number policy.
\setDocChangeRecord{%
  \addtohist{1}{YYYY-MM-DD}{Unreleased.}{William O'Mullane}
}


\begin{document}

% Create the title page.
%\maketitle
% Frequently for a technote we do not want a title page  uncomment this to remove the title page and changelog.
\mkshorttitle


\section{Introduction}

This note is a summary of work done for the Long Haul Network to USDF teats done on \jira{PREOPS-3571}.
This work was carried out in October 2023 - there may be more tests later.

\section {Test Setup}

To simulate as closely as possible the camera setup up eight of the camera nodes on the summit
were used to transmit data.

\subsection{Data}
Each node had some full focal plane image data on the local disk.
The chosen images were the pinhole images taken in 2020 since they have high information content and compress less than say flats from TS8.
The following  were provided by Tony Johnson :\\
\begin{tabular}{l l}
Romanesco  & MC\_C\_20200822\_000075\\
Vera\_Rubin & MC\_C\_20200822\_000054\\
Flammarion & MC\_C\_20200822\_000082\\
Group\_Photo & MC\_C\_20200822\_000061\\
\end{tabular}

These images may also be found in {\tt LSSTCam}  butler repo on USDF  using:
\begin{verbatim}
  exposure.obs_id IN ('MC_C_20200822_000075', 'MC_C_20200822_000054', 'MC_C_20200822_000082', 'MC_C_20200822_000061')}
\end{verbatim}

\subsection{Sending  and Scripts}

All scripts may be found in \url{https://github.com/lsst-dm/s3daemon}

A pooled connection S3 writer called {\tt s3daemon} was provided for K.T. Lim.
This was run on the background on each node.

The test client  {\tt send.py} can send one file and store it in S3 at the USDF with the given key.

The test scripts are einteh summittests folder of the s3daemon repo.
All scripts rely on a set of environment variables set in {\tt envvars.sh} this is not checked in but {\tt envvars_template.sh} gives the variables minus the actual S3 keys.

{\tt runs3Daemon.sh} runs the daemon and sets up a log file using the hostname.
 {\tt runOnNodes.sh .\runs3daemon.sh} is used to run that on all 8 nodes .

The {\tt sendTest.sh} uses {\tt send.py} to send 20 file simultaneously from the Test folder set up by Tony.
 {\tt runOnNodes.sh .\sendTest.sh} is used to run that ton all 8 nodes simultaneously.






\appendix
% Include all the relevant bib files.
% https://lsst-texmf.lsst.io/lsstdoc.html#bibliographies
\section{References} \label{sec:bib}
\renewcommand{\refname}{} % Suppress default Bibliography section
\bibliography{local,lsst,lsst-dm,refs_ads,refs,books}

% Make sure lsst-texmf/bin/generateAcronyms.py is in your path
\section{Acronyms} \label{sec:acronyms}
\addtocounter{table}{-1}
\begin{longtable}{p{0.145\textwidth}p{0.8\textwidth}}\hline
\textbf{Acronym} & \textbf{Description}  \\\hline

DM & Data Management \\\hline
LHN & long haul network \\\hline
OPS & Operations \\\hline
RTN & Rubin Technical Note \\\hline
S3 & (Amazon) Simple Storage Service \\\hline
SLAC & SLAC National Accelerator Laboratory \\\hline
USDF & United States Data Facility \\\hline
\end{longtable}

% If you want glossary uncomment below -- comment out the two lines above
%\printglossaries





\end{document}
